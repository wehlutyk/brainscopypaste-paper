% !TEX root = brainscopycut.tex
\section{Discussion}

\begin{new}

We aimed to connect the field of cultural evolution with psycholinguistics by asking if cultural attractors appear in a corpus of online news-related quotes which are gradually transformed by low-level biases.
The data set we used imposed a few constraints on our analysis:
first, it was necessary to infer source-destination links, an operation made more reliable when restricting the scope of transformations to very simple cases, which we did by focusing on word substitutions.
Second, contrary to laboratory experiments which produce data made of many repeated measures on a small number of cases (\hbox{e.g.} a given list of words), we have a great number of different cases (one case per cluster in which substitutions are found, \hbox{i.e.} 698 cases), with very few measures on each of them (average 9, median 5).
This rendered the prediction of individual words impractical: if we cannot compute a percentage of explained data for a given case, any approximate prediction will be heavily underestimated.
This last factor, added to the potential for variation of external conditions when authors wrote the quotes, led us to use word features to analyze the transformations by aggregating over individual cases.

By characterizing substitutions with 6 features on the disappearing word, we show that authors preferentially substitute words known for being harder to recall:
most prominently words with low frequency~\citep{gregg_word_1976}, learned later~\citep{dewhurst_separate_1998}, or made up of more letters~\citep{nickels_dissociating_2004}, both globally and in comparison to the sentence they appear in.
Further characterizing the substitutions by examining the variation of word features from disappearing to appearing words, we show:
(1) that the operation is contractile on average, that is words are brought closer to an attractor point on each feature;
(2) that authors produce words that are easier to remember than the average of synonyms of the disappearing word (a fact that is reflected in the position of the attraction point).

We do not actually observe quotes converging on a global scale towards attractors in their various dimensions.
Indeed the limits of the data set don't allow us to infer chains of substitutions, and substitutions themselves are not the only type of transformation at work in the data set.
Nonetheless, these findings (1) bring light to this simple type of transformation, and (2) are consistent with known psycholinguistic effects, with the hypothesis of cultural attractors in representations from everyday life, and with the lineage specificity discussed in the iterated learning literature~\citep{claidiere_cultural_2014,cornish_systems_2013}.
They are obtained by successfully applying knowledge from cognitive science to real-life complex data, a task that remains a challenge in the study of cultural evolution.
We believe that applying such data mining tools to manage the complexity of real-life data is a promising approach for the joint analysis of cognitive science and culture.

In the simple case presented here, however, much remains to be explored.
Since it is clear that observing cognitive biases in such data is now possible, questions addressed in controlled laboratory situations could be opened by further research.
One question concerns the influence of the context surrounding a quote, be it in terms of other quotes preceding it temporally or of text surrounding it in a post.
A first step could be the application of results from \citet{zaromb_temporal_2006} who have shown, in the simpler task of recall of random word lists, that the source of prior-list intrusions can be predicted based on the associations those preceding lists have formed:
in our case, a substitution could be triggered and directed by associations formed by preceding context.
A further step would be to follow \citet{cornish_systems_2013} have shown about reciprocal influences between context and transformations (in their case, with transmission chains of artificial content).
Indeed substitutions, and more generally all transformations, also participate in creating the context for later quotes.
One can ask, therefore, what are the reciprocal effects between, on one side, the corpus-level evolution of quotes through iterated transformations, and on the other side, a gradual change in the properties of transformations operated because of the evolution of surrounding context.
Such interactions have been shown to underlie the lineage specificity observed in transmission chains~\citep{claidiere_cultural_2014}.
Exploring how similar loop interactions happen real-life data could indeed be the next step in understanding the evolution of both cultural content and the ways in which it is transformed.
In our particular case, such insight could shed some light on how the feature attractors explored in this paper actually emerge.

\end{new}

% A few more unredacted ideas:
% More broadly with Sperber:
%  - you can have a simplistic interpretation and expect simple convergence, or you can look for attraction on different dimensions, which is a richer (and more useful) interpretation.
%  - note also that it can apply to long term or short term evolution.
%  - What we see is not clear w.r.t. convergence: (1) we don't observe it on our data, and (2) if there's lineage specificity there's no reason convergence should appear (on contrary). But, this is only half the story, since in reality there's also selection, i.e. not all chains go on forever, so you lose some diversity, and it changes the dynamics.
%  - finally, context is addressed with relevance in Sperber, but it's still all representational, as we did here. Others have said that it will never be enough. Ingold, Di Paolo.


\section{Concluding remarks}\label{sec:conclusion}

\newtext{The theory of Epidemiology of Representations proposes a unifying framework for the study of cultural evolution.
One of its core claims, the existence of cultural attractors, has been both a challenge to test empirically and a fruitful line to pursue in the study of cultural evolution.}
We aimed to contribute to \newtext{testing this hypothesis} by studying a simple everyday-life task where individuals are implicitly trying to reproduce quotations.
To some extent, our work amounts to \newtext{an out-of-laboratory} experiment where we \newtext{examine the influence} of \newtext{well-known word features} on the accuracy of the reproduction of short sentences.
\newtext{Our analysis of substitutions shows that words are attracted, in each dimension, to a feature-specific value.
Furthermore, the features' known effects in psycholinguistic experiments are reflected in the biases of these attraction points, meaning that the evolution of such quotations can be partially explained by known low-level cognitive biases.
We believe that such an approach, combining psycholinguistic knowledge and data mining tools, can be fruitfully developed to improve the study of cultural attractors and explore the reciprocal influences of cognition and culture.}

\begin{new}

Let us conclude by noting that the question of short- and long-term cultural evolution, and the approaches to study them, are becoming increasingly relevant to other fields.
In biology in particular, work on evo-devo and non-genetic inheritance has accumulated evidence that is poorly accounted for by the modern synthesis of biological evolution, and is creating a demand for new or extended  approaches to joint cultural and biological evolution~\citep[see][for instance]{gilbert_eco-evo-devo:_2015}.
Such an approach has long been called upon by anthropologists like~\citet{ingold_beyond_2004,ingold_three_1999}, in line with Mauss' initial works~\citep{mauss_les_1936}, and the debate is not entirely foreign to the enactive-representational discussion in cognitive science.
As regards our particular case, a most promising line of work on the evolution of language is being developed by \citet{cuffari_participatory_2014}.
The study of cultural evolution will most likely benefit greatly from the growing interactions between these disciplines.

\end{new}