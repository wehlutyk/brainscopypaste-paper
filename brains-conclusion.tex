% !TEX root = brainscopycut.tex
\section{Concluding remarks}\label{sec:conclusion}
We aimed at contributing to the empirical understanding of knowledge transformation processes by studying a simple task where individuals are \emph{implicitly} trying to reproduce textual content. To some extent, our work amounts to a large \emph{in vivo} experiment where we appraise the impact of classically-influent psycholinguistic variables in the accuracy of the reproduction.
In more detail, we describe the joint properties of the substituted and substituting terms in the reformulation by individuals of a specific type of utterances (quotations). %--- in this sense we also diverge from psycholinguistic experiments that focus on ease of recall. 

For each of the selected psycholinguistic variables, we demonstrate the existence of attractor values in the underlying variable spaces. More precisely, beyond the interpretation of our results for each variable, we notice that all variables remarkably exhibit a single attractor and are generally contractile --- as such, even though the observed convergence patterns only partially explain quotation evolution, we shed light on a class of phenomena which are prone to constitute a key element of a broader empirically-grounded, attractor-based theory of cultural evolution. 

%Rather, we exhibit the bias of substitution, and that in this respect it not only provides a  we provide a fine description of the bias but also corresponds to an  ``input-output'' reformulation couple describing the joint properties of (substituted$\rightarrow$substituting) terms.

%We contend that our results provide some of the first bricks of an empirical \emph{fitness landscape} for the epidemiology of representations

%(age of acquisition, number of phonemes, ), it also emphasizes the importance of the semantic network structure (Wordnet: Pagerank, degree, clustering[, distance?]), 


%\begin{itemize}
%\item new in the sense that it does not focus 
%From a psycholingustic viewpoint, 
