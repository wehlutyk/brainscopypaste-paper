% General document style ------------------------------------------------------------------
%\documentclass[jou]{apa6}
\documentclass[man]{apa6}
\usepackage[utf8]{inputenc}
\usepackage[T1]{fontenc}
\usepackage{enumerate}

% Colors
\usepackage[usenames,dvipsnames]{color}

%\acmVolume{VV}
%\acmNumber{N}
%\acmYear{YYYY}
%\acmMonth{Month}
%\acmArticleNum{XXX}
%\acmdoi{10.1145/XXXXXXX.YYYYYYY}

%\markboth{S. Lerique and C. Roth}{How do we copy and paste? The semantic drift of quotations in blogspace}

% Let us have multiline comments
\usepackage{verbatim}

% Footnotes in section headings
\usepackage[stable]{footmisc}

% Formatting bibliography
%\usepackage{natbib}
\usepackage[natbibapa,nodoi]{apacite}
%\setcitestyle{authoryear,round,semicolon}

% Mathematics and Code
\usepackage{amsmath}
\usepackage{bm}
\usepackage{amsfonts}
\usepackage[group-separator={,}]{siunitx}
%\usepackage{stmaryrd}
\usepackage{xfrac}
\usepackage{mathtools}
%\newcommand{\defeq}{\overset{\underset{\mathrm{def}}{}}{=}}

% Strikethrough (with the \sout comment)
\usepackage[normalem]{ulem}

\usepackage{listings}
\lstset{
    basicstyle=\ttfamily
}

% Have graphics
\graphicspath{{images/}}
\usepackage{graphicx}
%\usepackage{subfig}
%\usepackage{subfigure}
\usepackage{caption}
\usepackage{subcaption}

% Help in placing the floats
\usepackage{placeins}

% shortcuts
\newcommand{\arrival}{\text{arrival}}
\newcommand{\start}{\text{start}}

\newcommand{\warrival}{w'}
\newcommand{\wstart}{w}

% To make comments, corrections, notes
\newcommand{\tb}[1]{\textcolor{blue}{#1}}
\newcommand{\tbn}[1]{\tb{\sout{#1}}}
\newcommand{\tg}[1]{\textcolor{MidnightBlue}{#1}}
\newcommand{\newtext}[1]{\tg{#1}}
\newenvironment{new}{\par\color{MidnightBlue}}{\par}
\newcommand{\add}[1]{\tb{ADD: #1}}
\newcommand{\ra}[1]{$\rightarrow$#1}
\newcommand{\replace}[2]{\sout{#1} \tb{#2}}
\newcommand{\rk}[1]{\tb{{\footnotesize {\bf[\emph{#1}]}}}}
\newcommand{\marginnote}[1]{\marginpar{{\parbox{1.\linewidth}{\hrulefill\\\tb{\scriptsize \sf #1}}}}}
\newcommand{\cam}[1]{\tb{\small{[{\bf C:} }#1{]}}}
\newcommand{\seb}[1]{\tb{\small{[{\bf S:} #1{]}}}}
\newcommand{\todo}[1]{\tb{#1}}
\definecolor{lightblue}{rgb}{.85,.85,1}
\definecolor{lightgreen}{rgb}{.85,1,.85}

% If citations or references are missing, make it visible
\newcommand{\CN}{\textsuperscript{\tb{[Citation needed]}}}
\newcommand{\CNs}{\textsuperscript{\tb{[Multiple citations needed]}}}
\newcommand{\RN}{\textsuperscript{\tb{[Internal reference needed]}}}

% Start with the real content -------------------------------------------------------------

\title{The semantic drift of quotations in blogspace: a case study in short-term cultural evolution}
\shorttitle{The semantic drift of quotations in blogspace}
\date{}
\twoauthors{Sébastien Lerique}{Camille Roth}
\twoaffiliations{Centre d'Analyse et de Mathématique Sociales, UMR 8557 CNRS/EHESS\\190 av. de France, F-75013 Paris\\and Centre Marc Bloch Berlin, UMIFRE 14 CNRS/MAEE/HU\\Friedrichstr. 191, D-10117 Berlin}{CNRS\\Centre Marc Bloch Berlin, UMIFRE 14 CNRS/MAEE/HU\\Friedrichstr. 191, D-10117 Berlin}
\abstract{
\newtext{We present an empirical case study which connects psycholinguistics with the field of cultural evolution, in order to test for the existence of cultural attractors in the evolution of quotations.
Such attractors have been proposed as a useful concept for understanding cultural evolution in relation with individual cognition, but their existence has been hard to test.
We focus on the transformation of quotations when they are copied from blog to blog or media website:
by coding words with a number of well-studied lexical features, we show that the way words are substituted in quotations is consistent (1) with the hypothesis of cultural attractors, and (2) with known effects of the word features.
In particular, words known to be harder to recall in lists have a higher tendency to be substituted, and words easier to recall are produced instead.
Our results support the hypothesis that cultural attractors can result from the combination of individual cognitive biases in the interpretation and reproduction of representations.}
}
\authornote{Correspondence should be directed to lerique@cmb.hu-berlin.de and roth@cmb.hu-berlin.de}
\keywords{\newtext{word recall}; recollection bias; semantic network; cultural evolution; cultural attraction; data mining; big data; psycholinguistics}
%\note{\TB{Note XXXX (version? date?)}}

\begin{document}
\maketitle

% !TEX root = brainscopycut.tex

% ============================
\section{Introduction} % =====
% ============================

The understanding of knowledge transmission mechanisms has led to a sizable literature in the recent past, spanning over numerous research fields ranging from cultural anthropology to social network analysis and complex systems modeling; from social cognition to data mining. These works are diversely labeled as studies on ``opinion dynamics'', ``cultural evolution'', or ``information diffusion'' and, for the most part, investigate phenomena pertaining to both cognitive science and social science, both at the individual and social levels.

Broadly speaking, we may distinguish two main research streams, depending on whether the focus lies on cognitive processes or on social dynamics. A first stream is largely structured around cultural anthropology and essentially addresses cultural similarity, diversity and its evolution. It features several theories mixing social and individual cognition including, to cite a few, the debated ``memetic'' program initiated by \citet{Dawkins76} (for which the collection of works by \citet{Aunger00} provides a solid overview), the development of evolutionary models of norms (see for instance \citet{Ehrlich05}) following the seminal work of \citet{Boyd85}); or the ``cultural epidemiology'' program proposed by \citet{sper:expl}, which links the concept of mental representation to the concept of public representation (the latter being the counterpart of the former outside of the brain, \hbox{i.e.} in all sorts of cultural artifacts: texts, utterances, etc.).
%\footnote{\citet{sper:expl} emphasizes this distinction in his seminal work: ``A representation may exist inside its user: it is then a \emph{mental representation}, such as a memory, a belief, or an intention. The producer and the user of a mental representation are one and the same person. A representation may also exist in the environment of its user, as is the case, for instance, of the text you are presently reading: it is then a \emph{public representation}''.}

One of the core claims of this literature consists in emphasizing that not all knowledge is equally fit for being reproduced, although the various approaches have a different take on how exactly this notion of fitness should be operationalized. Sperber's cultural epidemiology classically opposes Dawkins' memetics by insisting that representations are not being replicated through a high-fidelity copy process, but are being interpreted and produced anew, and are thus greatly subject to change.
Cultural epidemiology postulates that this conceptual evolution can be appraised through the notion of ``cultural attractor'', seen as the attraction domain of an underlying socio-semantic dynamical system.\footnote{Works such as \citet{Atran03} argue that this approach is anthropologically better suited than memetics, and some of the main issues in this debate are further detailed by \citet{Kuper00} and \citet{Bloch00}.}
Despite some recent modeling attempts (\hbox{e.g.} \citet{Claidiere07}), the development of quantitative measurements relying on the concept of cultural attractors has remained a relatively hard task and, to our knowledge, this hypothesis has not yet been empirically analyzed in an extensive manner.


Another research stream deals with rather macroscopic studies of knowledge diffusion. Here, one of the focal points is that not all knowledge gets propagated identically along the same routes, within the same communities, at the same pace. The various approaches usually feature a minimalistic description of cognitive processes, strongly reminiscent of biological epidemiology (a single, atomic piece of information may or may not be adopted by each individual).  
This research program nonetheless exhibits a particularly interesting empirical track record --- largely owing to a recent avalanche of observable \emph{in vivo} data which, for a good decade now, have mainly come from online interaction contexts.
While these information trails are not records of ``physical'' inter-individual interactions (in the sense of ``real life'' interactions), they still constitute a wealth of observations on the dynamics of public --~albeit online~-- representations.
Some authors could describe for instance the propagation of cultural artifacts across social networks such as blogspace \citep{Gruhl04}, the characteristic times and diffusion cycles both within these social networks and with respect to the topical dynamics of news media \citep{Leskovec09}, or the reciprocal influence between the social network topology and the distribution of issues \citep{Cointet09}.

These latter studies are at the interface between data mining, complex systems and quantitative sociology (first and foremost social network analysis) and are relatively remote from cognitive science; for a significant part, they rely rather marginally on specific social science theories. They nonetheless show us the added value of using these rapidly growing records %--~given the already significant and rapidly growing importance of our online interactions~--
%and, %more broadly, large textual corpora \cam{google n-grams??} 
towards radically improving the empirical understanding of (individual-level) cultural evolution processes.


\bigskip
Stepping back, we thus observe a gap between, on one side, empirical studies of diffusion dynamics in social systems and, on the other side, more theoretical works  focused on knowledge transformation processes. %--- these latter studies being either strongly normative, or with results difficult to articulate with realistic cultural epidemiology models.
Our research lies at the intersection of these two programs, aiming to shed light on micro-level information transformation by leveraging the empirical wealth of (\emph{in vivo}) social diffusion phenomena. More precisely, we hope to describe reformulation processes within a large distributed system such as blogspace; showing how some specific types and features of public representations may be altered by bloggers when they freely reproduce them.

We focus on simple linguistic modifications, thereby connecting our research to the broader psycholinguistic literature.
To deal with robust and simple cultural representations, we paid attention to the evolution of quotations.
While these verbatim public representations should in theory not suffer any alterations when they are produced anew (as opposed to more elaborate expressions and opinions, not identified as quoted utterances), empirical observation shows that they are occasionally transformed.
We will in particular exhibit a non-trivial process by which individual words in quotations are replaced.
We will uncover some of the semantic and structural characteristics of these words and the substitutions they undergo.
In a way using this type of data is equivalent to a large-scale psycholinguistic experiment and at the same time constitutes a first step towards building empirically realistic models of cultural evolution.

\TB{The next section %(Sec.~\ref{sec:related}) 
describes our hypotheses along with the relevant state-of-the-art on this psycholinguistic matter.
%In Sec.~\ref{sec:protocol}, 
In the section that follows it, 
we detail the empirical protocol and the various assumptions that were made in order to deal with the available empirical material.
%Sec.~\ref{sec:results} 
The results section describes the significant psycholinguistic biases observed during \emph{in vivo} quotation reformulation as well as their epidemiological setting, followed by a discussion and general guidelines for further work in the final section.%Sec.~\ref{sec:conclusion}.
}

% ============================
\section{Related work} % =====
% ============================
\label{sec:related}

%The relevant literature on \emph{public representation dynamics} features two main streams.
%On one hand, we find studies of the macroscopic \emph{social diffusion} of public representations, describing for instance the propagation of cultural artifacts across social networks such as blogspace \citep{Gruhl04}, the characteristic times and diffusion cycles both within these social networks and with respect to the topical dynamics of news media \citep{Leskovec09}, or the reciprocal influence between the social network topology and the distribution of issues \citep{Cointet09}.
%These studies are relatively independent from anthropology and cognition and are at the interface between data mining, complex systems and quantitative sociology (first and foremost social network analysis).
%Without necessarily relying on specific social science theories, this research stream is of interest for its use of large textual corpora in studying cultural dynamics.
 
The practical study of the transformation of public representations has emerged only recently.
For one, models involving evolution and representations to study the notion of ``cultural attractor'' have appeared only a few years ago (see \citet{Claidiere07,claidiere2014darwinian} as well as a hybrid empirical-theoretical protocol in \citet{maccallum2012evolution}).
Among the empirical approaches, some of the most relevant studies to date consist in a series of papers investigating \emph{quotation} transformations in a large corpus of US blog posts, initially collected and studied by \citet{Leskovec09} and further analyzed by \citet{Simmons11} and \citet{omod-mult}.
%They exhibit several types of regularities and propose diffusion-transformation models of the evolution of quotations, which may nonetheless appear to be relatively simplistic from a cognitive viewpoint. One of the main observations in these works is that even for quotations, a type of public representation that should be among the most stable, it is still possible to witness significant transformations. However, these studies address transformations by focusing on the properties of the source of the quotation (\hbox{e.g.} news outlet {vs.} blog), or the surrounding public space (\hbox{e.g.} quotation frequency in the corpus), rather than the very cognitive-level features which may determine or, at least, influence these transformations.
One of the main observations in these works is that even for quotations, a type of public representation that should be among the most stable, it is still possible to witness significant transformations. They essentially examine the effect of some properties of the quotation source (\hbox{e.g.} news outlet {vs.} blog) or of the surrounding public space (\hbox{e.g.} quotation frequency in the corpus). Some diffusion-transformation models have been proposed, yet the very cognitive features which may determine or, at least, influence these transformations, are overlooked; which may appear to be relatively unsatisfying from a cognitive viewpoint.


At this level, we have to turn to the broader psycholinguistic literature which provides one of the main cognitive foundations for public representation evolution by studying the influence of word features on the ease of recall.
This field is well developed and details the impact that classical psycholinguistic variables such as word frequency (see \citet{Yonelinas02} for a review), age-of-acquisition \citep{Zevin02}, number of phonemes or number of syllables (see for instance \citet{Rey98} and \citet{nick-diss}), have in this type of task.

Less classical linguistic variables, based on the study of semantic network properties, have recently started to be used, in the context of connectionism and its normative processual models (see for instance \citet{collins1975spreading}).
Let us mention four interesting studies on that matter, which demonstrate in a strictly \emph{in vitro} framework and at the vocabulary level that properties computed on a word network are important factors for the cognitive processes and reproduction of those words.
First, \citet{Griffiths07} analyze a task where subjects are asked to name the first word which comes to their mind when they are presented with a random letter from the %Latin
alphabet. The authors show that there exists a link between the ease of recall of words and one of their semantic features, namely their authority position (pagerank) in a language-wide semantic network built from external word association data.
\citet{austerweil2012human} further develop this idea by showing that random walk on such a semantic network, i.e. the exact process measured by the pagerank index, gives a parsimonious account of some semantic retrieval effects (namely, related items being retrieved together).
A third psycholinguistic study by \citet{Chan10} shows, in a picture-naming task, that words are produced faster and with fewer mistakes when they have a lower clustering coefficient in an underlying phonological network (which, again, is  defined from external phonological data).
\citet{nelson2013activation}, finally, show the importance of clustering coefficient in a semantic network by studying the role it plays in a variety of recall and recognition tasks (extralist and intralist cuing, single item recognition, and primed free assocation).

On the whole, the current psycholinguistic state-of-the-art seems to hint towards two antagonistic types of results.
On one hand, part of the literature tends to show that recall is easier for the least ``awkward'' words; those whose age of acquisition is earlier, length is smaller, semantic network position is more central -- this is particularly true in tasks where participants are asked to form spontaneous associations or utter a word in response to a given signal.
On the other hand, when the task consists in recognizing a specific item in a list, ``awkward'' words are actually more easily remembered, possibly as they are more informative and plausibly more discernible (see again~\citet{Yonelinas02} for a review).
The jury is still out as to whether reformulation alteration, i.e. spontaneous replacement of words when asked to repeat a given utterance, is rather of the former or latter sort.
We also aim here at shedding some light on this debate, considering oddness as a dimension of the purported fitness of utterances.


% !TEX root = brainscopypaste.tex
% ============================
\section{Protocol} % =========
% ============================
\label{sec:protocol}

In order to start bridging this gap, we set out to \emph{empirically} study public representation transformations at the microscopic level, aiming to stay compatible with macroscopic-level studies of these public representations.
Quotations appeared to be a perfect candidate as public representations.
First, they are usually cleanly delimited by quotation marks (and often with HTML markup in web pages), which greatly facilitates their detection in text corpora.
Second, they stem from a unique ``original'' version, and could ideally be traceable back to that version.
Third, and most importantly, their duplication should \emph{a priori} be highly faithful, apart from cases of cropping: not only should transformations be of moderate magnitude, but when specific words are not perfectly duplicated, it is safe to assume that the variation is due to involuntary cognitive bias --- as writers may expect any casual reader to easily verify, and thus criticize, the fidelity to the original quotation.
Quotation evolution is therefore a perfect environment to measure cognition-induced transformations and relate those findings to macroscopic social dynamics.

\subsection{Dataset}

We used a reliable quotation dataset collected by \citet{Leskovec09}, large enough to lend itself to statistical analysis.
This dataset consists of the daily crawling of news stories and blog posts from around a million online sources, with an approximate publication rate of 900k texts per day, over a nine-month period of time (from August 2008 to April 2009) \cite{Leskovec09-url}.\footnote{Unfortunately, the original article~\citep{Leskovec09} does not provide additional details on the source selection methodology.}
Quotations were then automatically extracted from this corpus: each quotation is a more or less faithful excerpt of an utterance (oral or written) by the quoted person. \cam{btw do we have an excerpt of quotations here, could we even feature a few prototypical examples?}
Quotations were then gathered in a graph and connected according to their similarity: either because they differ by very few words (in that case, no more than one word) or because they share a certain sequence of words (in that case, at least ten consecutive words).
A community detection algorithm was applied to that quotation graph to detect aggregates of tightly connected, i.e. sufficiently similar, groups of quotations (see \citet{Leskovec09} for more detail).
This analysis yielded the final data we had access to, with a total of about \num{70000} sets of quotations; each of these sets allegedly contains all variations of a same parent utterance, along with their respective publication URLs and timestamps.

\subsection{Word-level measures}

To keep the analysis palatable, we restricted the analysis to quotation transformations which consisted in the \emph{substitution} of a word by another word (and only those cases) in order to unambiguously discuss single word replacements. 
To quantify those substitutions, we decided to associate a number of features to each word, the variation of which we can statistically study.
The following sections detail the features we used.

\subsubsection{Standard psycholinguistic indices}

We first introduce some of the most classical psycholinguistic measures on words:

\rk{Add some bibliography about those features' known effects}

\begin{itemize}
    \item \textbf{Word frequency}: the frequency at which words appear in our dataset, \rk{add ref on why it's important}
    \item \textbf{Age of Acquisition}: the average age at which words are learned, obtained from~\citet{kuperman12},
    \item The average \textbf{Number of Phonemes} for all pronunciations of a word, obtained from the Carnegie Mellon University Pronouncing Dictionary~\citep{Weide98},\footnote{The CMU Pronouncing Dictionary is included in the NTLK package~\citep{Bird09}, the natural language processing toolkit we used for the analysis.}
    \item The average \textbf{Number of Syllables} for all pronunciations of a word, also obtained from the CMU Pronouncing Dictionary,
    \item The average \textbf{Number of Synonyms} for all meanings of a word, obtained from WordNet~\citep{WordNet10}. \rk{add ref on why it's important}
\end{itemize}

We also considered grammatical types within quotations by detection of \emph{Part-of-Speech} (POS) categories, using the Penn TreeBank Project typology~\citep{Santorini90} and thereby distinguishing verbs, nouns, adjectives and adverbs.
The results were however extremely similar across the various categories, exhibiting no specific effect of words belonging to different POS categories.
\rk{See \#8 for this fact-check}

\subsubsection{Network-based measures}

Aside from classical psycholinguistic measures, we also considered more recently studied variables based on semantic network properties.
We relied on the ``free association'' norms collected by~\citet{Nelson04} which naturally embed information on the idea association process underlying transformation of quotations.

Free association (FA) norms record the words that come to mind when someone is presented with a given cue (that is the ``free association'' task).
As \citeauthor{Nelson04} explain,
\begin{quote}
free association response probabilities index the likelihood that one word can cue another word to come to mind with minimal contextual constraints in effect.~\citep{Nelson04}
\end{quote}
%Following \citet{Griffiths07}, we first consider the directed weighted network formed by the association norms, that is the network where words are nodes and edges are directed from cue to associated word, with a weight equal to the probability of that target word being produced when this particular cue was presented.
% \rk{we're in fact using the unweighed version of the network. Why?}
\new{Following \citet{Griffiths07}, we first build a directed unweighted network based on association norms, where words are nodes and edges are directed from cue to target word whenever a target word is being produced when this particular cue word was presented.}
This network is of particular interest since it measures the \emph{in-vitro forced-choice} version of a substitution whereas the data we analyse is the \emph{in-vivo spontaneous} version of what we otherwise hypothesize to be the same process.

\bigskip
We introduce three standard network-based measures to be used on the FA network% \rk{adapt once the weighing question (\#8) is settled}
:

\begin{itemize}
    \item \textbf{Centrality} $k$, initially measured by the number of incoming edges to a given node, i.e. the number of cues for which a given word is triggered as an association, which strongly relates to word polysemy.
    However in the present case there is a quasi-perfect correlation between node incoming degree and node \emph{pagerank}~\citep{Page99}, which will lead us to favour the latter later on. Word pagerank on the FA network had already been used by~\citet{Griffiths07}; it may be interpreted as a generalized and recursive measure of word polysemy: central nodes in the pagerank sense are words often selected as targets when presented with cues themselves often selected as targets, and so on recursively.
    \item \textbf{Clustering coefficient} $c$, which measures the extent to which a node belongs to a local aggregate of tightly connected nodes, and defined as the ratio between the number of actual \emph{v.} possible edges between a node's neighbours \cite{watt-coll}.
    We compute the clustering coefficient on the undirected version of the FA network; we thus measure if a word belongs more or less to a local aggregate of equivalent words (from a ``free association'' point of view).
    \item \textbf{Betweenness coefficient} $b$, another measure of node centrality describing the extent to which a node tends to connect otherwise remote areas of the network~\citep{free:set}.
    More technically, it corresponds to the normalized number of shortest paths connecting dyads which pass through that node; the higher the coefficient, the more important that node is in ensuring the connectedness of the rest of the network.
    This quantity tells us if some words behave like unavoidable waypoints on the path associating one word to another.
\end{itemize}

\subsubsection{Variable correlations}

An important question arises concerning the possible correlations between all the variables we use.

\begin{figure}[!th]
    \centering
    \includegraphics[width=\linewidth]{images/computed-figures/feature_correlations-filter1.png}
    \caption{Spearman correlations in the initial set of features}
    \label{fig:feature-corrs-initial}
\end{figure}

\rk{Correlation talk needs to be redone for new set of features}

Age of acquisition is a key variable which appears as a usual suspect in psycholinguistic studies and is also usually correlated to many of the other variables.
This relates to an ongoing debate suggesting that age of acquisition encodes a variety of phenomena, difficult to disentangle from more specific phenomena which could be captured by more independent variables~\CN{}.
Here however, as can be seen in Figure~\ref{fig:feature-corrs-initial}, age of acquisition has a relatively low correlation to the other variables (absolute value not above $0.42$ if we exclude the centrality measures), leading us to keep the variable in the rest of the analysis.

Number of phonemes and number of syllables naturally exhibit a strong linear correlation ($0.83$).
The analysis showed a better prediction effect of number of phonemes over number of syllables, which is consistent with~\citet{nick-diss}, and we therefore chose to focus the presented results on the former only.

Frequency and number of meanings both have relatively low levels of correlation to the other variables; we therefore also keep them in the rest of the analysis.

\bigskip
Network properties, on the other hand, are strongly dependent on one another.
As mentioned earlier, word degree and word pagerank have a very strong correlation ($0.89$) and, degree being generally more correlated to other variables, we chose to remove this variable from the results presented.
Finally betweenness centrality also exhibits strong correlation levels to the other network properties ($0.62$, $0.64$, and $0.72$ in absolute value), leading us to drop this final feature due to its redundancy.

The final set of variables we consider, as well as their cross-correlations, can be seen in Figure~\ref{fig:feature-corrs-filtered}.

\begin{figure}[!th]
    \centering
    \includegraphics[width=0.6975\linewidth]{images/computed-figures/feature_correlations-filter2.png}
    \caption{Spearman correlations in the filtered set of features}
    \label{fig:feature-corrs-filtered}
\end{figure}


\subsection{Temporal binning}

The data we use presents an additional challenge: each set of quotations bears no explicit information either about the authoritative original quotation, or about the source quotation(s) each author inspired himself from when creating a new post and reproducing (possibly altering) those sources.
Quote-to-quote transformations, and much less substitutions, are therefore not explicitly encoded in the dataset.

We face an inference problem where, given all quotations and their occurrence timestamps, we should estimate which was the originating quotation for each instance of each quotation.
We therefore model the underlying quotation selection process by making a few additional assumptions which let us define quote-to-quote substitutions from the available data.
The main issue at hand is deciding whether a later occurrence is a strict copy of an earlier occurrence, or a substitution of an even earlier occurrence, or perhaps even a substitution or copy from quotes appearing outside the dataset, \hbox{i.e.} from a source external to the data collection perimeter.

Let us give an example: say the quotation ``These accusations are false and \textbf{absurd}'' ($q_a$) appears in a blog on January 19, and the slightly different quotation ``These accusations are false and \textbf{incoherent}'' ($q_b$) appears in other blogs on the 21st, 22nd and 23rd of January.
If $q_a$ was sufficiently prominent when $q_b$ first appeared, we can safely assume that the first author of $q_b$ on the 21st based himself on $q_a$ as is shown in Figure~\ref{fig:substitution-temporal-binning-a}.
But what about the second and third occurrences of $q_b$, on the 22nd and 23rd?
Should we consider them to be substitutions based on $q_a$ %(i.e. re-creations of $q_b$ by a new instance of the substitution process that brought from $q_a$ to $q_b$ in the first place) 
or accurate reproductions of the previous occurrences of $q_b$? (Options shown in Figure~\ref{fig:substitution-temporal-binning-a}.)

\begin{figure}[h]
    \centering
    \subfloat[Possible paths from occurrence to occurrence]{
	    \def\svgwidth{\linewidth}
	    \small
	    \input{images/substitution-temporal-binning-a.pdf_tex}
	    \label{fig:substitution-temporal-binning-a}
	}
	\hfill \\
    \subfloat[Binned quotation family]{
	    \def\svgwidth{\linewidth}
	    \small
	    \input{images/substitution-temporal-binning-b.pdf_tex}
	    \label{fig:substitution-temporal-binning-b}
	}
	\caption{Temporal binning of quotation families}
    \label{fig:substitution-temporal-binning}
\end{figure}

To settle this question we bin the quote occurrences into fixed \emph{time bags} spanning $\Delta t$ days (2 days in the implementation), each one representing a unit of time evolution.
Then when a quotation $q$ appears in time bag $n$, it is counted as a substitution from each quote $q^*$ in the preceding time bag ($n - 1$) from which it differs by only one word.
If no quote in the preceding time bag can qualify as a source in a substitution (i.e. $q$ differs from all the quotes in the preceding time bag by more than one word), the occurrence of $q$ is not considered to be an instance of substitution.
Such a model defines how many times quote occurrences can be counted as substitutions: in Figure~\ref{fig:substitution-temporal-binning-b}, occurrences of $q_b$ on the 21st and 22nd are counted as substitutions, whereas the occurrence on the 23rd is not.

The assumptions embedded in this model are only a subset of a wider set of possibilities, each leading to alternative substitution inferences.\footnote{In particular, time can be sliced into bins to build fixed time bags as is done here, or kept fine-grained by using sliding time bags.}
These various flavours of an ideal substitution detection model essentially change whether occurrences are considered as substitutions from another quote, repetitions of the original quote, or introduction of information external to the dataset.
We identified and implemented eleven other such models, and they all yielded essentially the same results.

\section{Macroscopic evolution of quotation families}

We first examine the evolution of quotation families under the repeated action of substitutions.
Our goal in this step is to identify the long-term effect of cognitive bias over the lifetimes of quotation families and thus over the framing of public information.

To do so we compute the distribution of word features for each time bag $\mathcal{B}_n$ of each quotation family, and sum those distributions over all quotation families.
This yields a distribution of feature values for each $n$, which is the simplest possible view of the state of an average quotation family in its $n$-th time bag, i.e. after $n \times \Delta t$ days.\footnote{For consistency, if we do this for $n$ going up to $N$, we only include quotation families that span long enough to have at least $N$ time bags.}
Such a computation, based solely on the binning of quotation families, makes no assumptions on the way quotations undergo substitutions over time.

The raw distributions built with this computation are stationary.
That is, the substitutions on quotations over time have no global effect on quotation families, either because external quotations are continuously fed into the family and compensate for the effect of substitutions, or because substitutions operate on a marginal portion of the quotation families, or both.

\bigskip
To narrow this view to the specific effects of cognitive bias we consider substitution \emph{chains}: using the substitution detection model described in the previous section, we filter time bags to include only new quotes produced by substitutions themselves based on quotes produced by substitutions, and so on recursively.
The first time bag is therefore untouched, the second contains the quotations produced by substitutions from the first, the third by substitutions from the second, and so on (see Figure~\ref{fig:substitution-chains} for an illustration of this process).
As a result, the number of observed words drastically decreases across time, yet it unambiguously focuses on successive mutations.

\begin{figure}[h]
    \centering
    \def\svgwidth{\linewidth}
    \tiny
    \input{images/substitution-chains.pdf_tex}
    \caption{XXXXX}
    \label{fig:substitution-chains}
\end{figure}

We thus observe only the repeatedly-affected portion of the quotation families, and obtain a simple view of the longitudinal effect of cognitive bias, \hbox{i.e.} how quote features are evolving in the long term and, perhaps, converging towards specific attractors.

\rk{Comment results and transition to micro process}

\begin{figure*}[!tbh]
    \centering
    \includegraphics[width=\textwidth]{images/computed-figures/timebags_evolution_recursive_min20_no-exclusion.png}
    \caption{\textbf{Feature distribution evolution:} evolution of the distribution of feature values in substitution chains over successive 2-day time bags (bags 0 to 18, i.e. days 0 to 37).
    The legends indicate the number of words left in each time bag; these decrease exponentially since only a fraction of the quotes undergo substitution at each step.
    After a period of time, each feature becomes concentrated in a specific range of its own.}
    \label{fig:timebags-evolution}
\end{figure*}

\section{Microscopic evolution: the substitution process}

We then focus on the individual substitution process at work when authors transform quotations, by examining the features of the substituted and substituting words in each substitution.
Note that since we only consider substitutions and not faithful copies, we measure the features of an alteration \emph{knowing that there has been an alteration}, and we do not take invariant quotations into account.
Indeed, in the first case we know there has been a human reformulation, whereas in the second case it is impossible to know whether there has been perfect human reformulation or simply digital copy-pasting of a source (``{\sc Ctrl-C}/{\sc Ctrl-V}'').

\TB{This model has two main effects.
First, the binning influences how many times quote occurrences can be counted as substitutions: in Figure~\ref{fig:substitution-temporal-binning-b}, occurrences of $q_b$ on the 21st and 22nd are counted as substitutions, whereas the occurrence on the 23rd is not.
Second, the ``majority'' rule defines how quotes are sourced in substitutions: in the third drawing on Figure~\ref{fig:substitution-temporal-binning}, $q_b$ holds the majority and is the considered the basis for the last occurrence of $q_a$, in spite of $q_a$ having appeared earlier at the very beginning (indeed in the situation shown in Figure~\ref{fig:substitution-temporal-binning}, this seems to be the most likely scenario).}

\TB{\textbf{Source quotes} in the source interval for substitutions can be further restricted, e.g. to only the most frequent quote in the interval as in the model described above,}

\begin{figure}[h]
    \centering
    \def\svgwidth{\linewidth}
    \small
    \input{images/substitution-q_max.pdf_tex}
    \caption{XXXXX}
    \label{fig:substitution-q_max}
\end{figure}

We build two main observables for each word feature.
First, we measure the susceptibility for words to be the source of a substitution, knowing that there has been a variation, in order to show which semantic features are the most likely to attract a substitution under this condition.
Second, we measure the variation of word feature over a substitution, looking at the variation of a given feature between start and arrival words.

\subsection{Susceptibility}

For a given feature $\phi$, the protocol lets us compute substitution \emph{susceptibilities} for each feature value $f$.
We say that a word is \emph{substitutable} if it appears in a quote which undergoes a substitution, whether that substitution operates on the considered word or on another.
Word substitution susceptibility is computed as the ratio of the number $s_w$ of times a word is substituted to the number $p_w$ of times that word appears in a substitutable position, i.e. $\sfrac{s_w}{p_w}$.

Now averaging over all words such that $\phi(w) = f$ (only taking into account words that are substituted at least once), we obtain the mean susceptibility for the feature value $f$:
\footnote{To avoid any auto-correlation effect due to the number of substitutions in a cluster (possibly leading to an overly optimistic estimation of confidence intervals), we first average substitutions over each cluster, by considering the average of arrival word features for a given start word.
Indeed, substitutions occurring in the same cluster are likely not statistically independent.}
$$\sigma_{\phi}(f) = \left< \frac{s_w}{p_w} \right>_{\left\lbrace w | \phi(w) = f \right\rbrace}$$

This measure focuses on the selection of start words involved in substitutions, measuring the effect of features at the moment preceding the substitution when it is not yet known which word in the quotation -- if any -- will be substituted.

\subsection{Alteration}

Next, we measure how a word $\wstart$'s feature varies as $\wstart$ is substituted by $\warrival$, i.e. $\phi(\warrival) - \phi(\wstart)$.
Averaging this value over all start words such that $\phi(w) = f$ yields the mean variation for that feature value~$f$:
$$\Delta_{\phi}(f) = \left< \phi(\warrival) - \phi(\wstart) \right>_{\left\lbrace (\wstart,\warrival) | \phi(\wstart) = f \right\rbrace}$$

We introduce a null hypothesis $\mathcal{H}_0$ to compare the actual variation of a word's feature to its expected variation, assuming the arrival word $\warrival_0$ had been chosen randomly from the pool of free association words.
The new quantity under $\mathcal{H}_0$ is:
\footnote{Note that $\phi(\warrival_0)$ is in fact a constant in this averaging, since by definition $\warrival_0$ does not depend on $\wstart$.}
$$\Delta_{\phi}^0 (f) = \left< \phi({\warrival_0}) - \phi(\wstart) \right>_{\left\lbrace (\wstart,\warrival_0) | \phi(\wstart) = f \right\rbrace}$$

We also considered an alternative null hypothesis, denoted $\mathcal{H}_{00}$, where the arrival word is chosen randomly \emph{among immediate synonyms of the start word}, i.e. an arrival word chosen among semantically plausible though still random words.
\footnote{In this case $\warrival_{00}$ does depend on $\wstart$.} \rk{Do we present these results?}\cam{nope}\rk{S: Then we must say the results are the same.}

Using this method we obtain the mean variation of feature for each start feature value, and can compare the variations to a situation where arrival words are chosen randomly.
This gives us a fine-grained view of how word features evolve upon substitution.

\subsection{Results}

\rk{Comment results}

\begin{figure*}[!th]
    \centering
    \includegraphics[width=\textwidth]{images/computed-figures/feature_susceptibilities.png}
    \caption{\textbf{Substitution susceptibility:} average susceptibility to substitution \emph{v.} average feature value of a candidate word for substitution, with 95\% asymptotic confidence intervals.
    Each feature exhibits a specific and significant pattern favouring either high- or low-valued words for substitution.}
    \label{fig:feature-susceptibilities}
\end{figure*}

\begin{figure*}[!th]
    \centering
    \includegraphics[width=\textwidth]{images/computed-figures/feature_variations-binned.png}
    \caption{\textbf{Feature variation upon substitution:} average feature of the appearing word minus $\mathcal{H}_0$ \emph{v.} average feature of the disappearing word in a substitution, with 95\% asymptotic confidence intervals.
    The overall position of the curve with respect to $y = 0$ indicates the direction of the cognitive bias.
    The fact that all the curves have slopes smaller than 1 means that the substitution operation is contractile on average: each feature will converge towards its own specific asymptotic range, which is consistent with the evolution observed in Figure~\ref{fig:timebags-evolution}.}
    \label{fig:feature-variations}
\end{figure*}

% !TEX root = brainscopycut.tex
\section{Discussion}

- place back into the initial question: in our example, do reps converge?
-> we don't know because we couldn't test (ref. sub models and complicated data), but there is attraction.
  - Remind the details
  - It's consistent with known effects (word frequency: easier recall for high freq, age of acquistion: Zevin, Dewhurst, Morrison, clustering coefficient: nelson, age of acquisition (corr pagerank): griffiths, orth. neighborhoos: Garlock), which is interesting, because it's applying cogsci tools on real life data
  - it's also a first step towards analyzing together cogsci and culture (because cogsci on real life data), which is much of a challenge still, although we don't get to see any global effect on the corpus nor reciprocity, therefore (i.e. we don't loop the loop yet)
  - there's a cost to this however: noise, uncertainty (hence single sub), data shaped oppositely to lab experiments.
  - so naturally it's a far cry from the whole story. It's not even the main transformation at work, so we can't conclude on the evolution of the dataset. But we know something more for this simple case, and what it costs to reach that knowledge.

- Now we also looked at context, but there's vast oceans to explore in this direction too. Indeed there's way more to do on this case:
  - in lab exps you can predict the word if proper measures or well-designed lists
  - you can ask what triggers the substitution with semantic similarity: low cost of switching to a more frequent dyad, given the sentence, following Zaromb
  - you can see how previous context influences the semantic similarity (PLI, LSA/LDA). Note that these LSA/LDA methods are ill-suited to the data.
  - with better data, building on our software, we could do that, and look at chains

More broadly with Sperber:
  - you can have a simplistic interpretation, or you can look for attraction on different dimensions, which a more rich (and useful) interpretation.
  - kirby and co have looked at that on artificial languages, on non-meaningful patterns (claidiere), we could look at that on real language.
  - What we see is consistent with lineage specificity (contraction around sentence median), and many things push towards that (context that influences the fitness, as in "language as a system" from kirby, previous history that changes bias, as in PLIs), but right now we don't know in detail what pressure creates that. It's likely the type of task influences that.
  - What we see is not clear w.r.t. convergence however: (1) we don't observe it on our data, and (2) if there's lineage specificity there's no reason convergence should appear (on contrary). But, this is only half the story, since in reality there's also selection, i.e. not all chains go on forever, so you lose some diversity, and it changes the dynamics.
  - finally, context is addressed with relevance in Sperber, but it's still all representational, as we did here. Others have said that it will never be enough. Ingold, Di Paolo.




















\todo{Discuss related to introduction. Attractors, lineage with specification, what we couldn't observe, how it fits into Kirby.
}



\todo{\#15: relate to missed literature}

\begin{new}

ADDTHIS:
We also chose exploratory vs. predictive to give a detailed view of what happens and because there's too many possible things to predict.

ADDTHIS:
By characterizing substitutions with 6 features on the disappearing and appearing words, we identify what makes a substitution more likely, and how a word changes when it is substituted.
Consistent with known effects in linguistics, we observe that low-frequency words and words learned later in development are more susceptible to substitution than other words.
Looking at the context those words appear in, we observe a marked effect for substitution of extreme words in a sentence (either very high-valued or very low-valued features compared to sentence average, except for word frequency).
Focusing on how words are transformed, we see that the appearing words have significantly higher frequency and lower age-of-acquisition than synonyms of the disappearing word.
Finally, the patterns we observe are also consistent with an attraction of each of the features towards a (feature-specific) asymptotic value.

ADDTHIS:
It is possible however, that these attractors appear due to an interaction between biases and sentence context, making it a contingency rather than a rule. This is not really dealt with (context, aside from relevance) by Sperber.

Attraction can also be defined on any number of dimensions. It can be on the structure, on anything, so saying there could be an attraction while not specifying the dimension is really meaningless. What's more important is to look at a specific dimension, and see if there's attraction on that one, as we did here for features.

We could've done also on semantic grouping, predicting the new word based on semantic similarity (or on frequent dyads, i.e. collocation with previous word the same way Zaromb et al. 2006 explain PLIs), and predicting the disappearing word based on the cost of doing such a substitution (lower cost -> higher prob of substitution). The point is, there's decades and many fields of psycholinguistics, and we can connect each of them with this question.

All of this is possible with our software that we published.

ADDTHIS:
Taking context into account is more than what we did. For instance, building on Zaromb 2006, you could imagine that the substitutions appear because the word preceding the substituted one appears in another dyad a lot more than this one, triggering a substitution (Zaromb's associative vs. contextual retrieval processes in recall).

ADDTHIS:
Soooooo... following literature on word lists (Zaromb and DRM):
- we could predict the new word in substitution? Take the strongest average association to words in the sentence.
- we could predict substituted word? Take the word following the one that triggers the new word.
Problems:
- we probably won't find the exact word, but one similar to it (even Zaromb can't predict the exact word, they don't try, they just check it comes from the right list). How to evaluate that?
- if computing LSA/LDA on the corpus (which probably isn't adapted because of the short sentence nature of the data -> topics = quote families), it's tautological unless you suppose substitutions have a negligible effect.
Explain that to the reviewer.

Again justify our approach w/ features by the shape of our data: few substitutions per cluster (avg. 9), and substitutions on relatively few clusters overall -> opposite situation to a few lists repeated through 50 subjects where frequency of transition has meaning. Here for each word, it's nearly one shot. So we have to categorize words (by using low number of features -> known features best) to get frequencies. From there, you can predict many many things, so better describe.


\end{new}

\section{Concluding remarks}\label{sec:conclusion}

\todo{\#14:
link to introduction discussion:
(1) this can be a model system,
(2) convergence can be looked for in any dimension, but that doesn't make a theory, so Epidemiology of Representations is all nice, but:
(3a) taken simplistically it predicts obviously simplistic stuff (all quotes CV. to a single quote),
(3b) taken more realistically (many dimensions in life) it gives some ideas, but it's not clear it's a core principle
(3c) we need more controlled investigation to fuel the discussion and see how relevant it is.
}

\begin{new}


ADDTHIS:
On the other side an enactive proposition which anthropologists like Ingold, in line with Mauss' works, are calling for \CN, is being developed by Froese, Di Paolo, and De Jaegher among others \CNs.

The question is also gaining relevance in other fields, as work in evo-devo and non-genetic inheritance is accumulating evidence not accounted for by the modern synthesis \CN;
these discoveries are creating demand for new or extended approaches to life evolution that unify its different levels, as well as creative empirical methods to test the predictions these approaches make \CN.

\end{new}

We aimed to contribute to the empirical understanding of representation transformation processes %\marginnote{fixed after Telmo saying: Is this really about knowledge transformation of more about representation transformations and the insight they provide over underlaying cognitive processes?}
 by studying a simple task where individuals are \emph{implicitly} trying to reproduce textual content. To some extent, our work amounts to a large \emph{in vivo} experiment where we appraise the impact of classically-influent psycholinguistic variables in the accuracy of the reproduction.
In more detail, we describe the joint properties of the substituted and substituting terms in the reformulation by individuals of a specific type of utterances (quotations). %--- in this sense we also diverge from psycholinguistic experiments that focus on ease of recall.

\todo{\#18: tone down claims about contractile: it's a possible hypothesis if this were the only process, but not observed with the mix of all other processes}

For each of the selected psycholinguistic variables, we demonstrate the existence of attractor values in the underlying variable spaces. More precisely, beyond the interpretation of our results for each variable, we notice that all variables remarkably exhibit a single attractor and are generally contractile --- as such, even though the observed convergence patterns only partially explain quotation evolution, we shed light on a class of phenomena which are susceptible to constitute a key element of a broader empirically-grounded, attractor-based theory of cultural evolution.

%Rather, we exhibit the bias of substitution, and that in this respect it not only provides a  we provide a fine description of the bias but also corresponds to an  ``input-output'' reformulation couple describing the joint properties of (substituted$\rightarrow$substituting) terms.

%We contend that our results provide some of the first bricks of an empirical \emph{fitness landscape} for the epidemiology of representations

%(age of acquisition, number of phonemes, ), it also emphasizes the importance of the semantic network structure (Wordnet: Pagerank, degree, clustering[, distance?]),


%\begin{itemize}
%\item new in the sense that it does not focus
%From a psycholingustic viewpoint,


\subsection*{Acknowledgements}

We are warmly grateful to Ana Sofia Morais for her precious feedback and advice on this research, and to Telmo Menezes, Jean-Philippe Cointet, Jean-Pierre Nadal, Sharon Peperkamp, \newtext{Nicolas Claidière} and Nicolas Baumard for useful suggestions and comments.

{\new This work has also been partially supported by the French National Agency of Research (ANR) through the grant Algopol (ANR-12-CORD-0018).}

\begin{new}

\subsection*{Software colophon}

Finally, this paper was developed using Python's scientific computing ecosystem~\citep{millman_python_2011}.
In particular, we directly used NumPy and SciPy~\citep{walt_numpy_2011}, Matplotlib~\citep{hunter_matplotlib:_2007}, Pandas~\citep{mckinney_data_2010}, scikit-learn~\citep{pedregosa_scikit-learn:_2011}, NetworkX~\citep{hagberg_exploring_2008}, NLTK~\citep{bird_nltk_2009}, IPython~\citep{perez_ipython:_2007}, and many other libraries from the Python ecosystem.
The software and analyses written for the paper are documented and published under a Free Software license.
They can be found at \url{https://github.com/wehlutyk/brainscopypaste}.

\end{new}

\bibliographystyle{apacite}
\bibliography{brainscopycut.bib}

\end{document}
