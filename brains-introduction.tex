% !TEX root = /Users/camille/Shared/Dropbox/BrainCopyPaste/brainscopypaste.tex
% ============================
\section{Introduction} % =====
% ============================


%On assiste aujourd'hui à une convergence progressive entre des disciplines étudiant la cognition humaine, d'un côté, et des disciplines étudiant la culture et les \emph{choses sociales}, de l'autre. Ces deux familles de disciplines ont en effet bien des frontières en commun ; celles-ci se rendent visibles par l'intermédiaire des thèmes pour lesquels il devient incontournable de s'intéresser à l'humain à la fois dans sa globalité et dans ses fonctionnements particuliers. 

The understanding of the origin of cultural similarity has triggered a sizable literature in the recent past, spanning over a vast area of research fields ranging from cultural anthropology to social network analysis and complex system modeling --- and diversely labeled as studies on ``opinion dynamics'', ``cultural evolution'', or ``information diffusion'', for the most part.  Broadly, this type of research program relies on the understanding of phenomena pertaining to both cognitive science and social science, in order to jointly appraise the processing and transmission of information at the individual and social levels. % information propagation phenomena -- and where we witness the development of studies relying on both individual and social cognition.

Several theories have been proposed and debated within these research fields, mixing social and individual cognition --- notably from the side of cultural anthropology.  There is, first, the debate around the ``memetic'' program initiated by \citet{Dawkins76}, for which the collection of works by \citet{Aunger00} provides a solid overview. A second field relates to the development of evolutionary models of norms (see for instance \citealp{Ehrlich05}), following the seminal work of  \citet{Boyd85}. 

A third approach, ``cultural epidemiology'' initiated by \citet{sper:expl} recently attracted significant attention. Works such as \citet{Atran03} suggest that this stance could be anthropologically more satisfying than memetics; see \citet{Kuper00}, and \citet{Bloch00}, for further details on the main issues in this debate.
It focuses on the notion of \emph{representation}, linking the cognitive science concept of \emph{mental representation} to the concept of \emph{public representation}, the latter being the equivalent of the former outside of the brain (\hbox{i.e.} in cultural artefacts: texts, utterances, etc.).\footnote{\citet{sper:expl} emphasizes this distinction in his seminal work:
%\begin{quotation}
{%Une représentation peut exister à l'intérieur même de l'utilisateur : il s'agit alors d'une \emph{représentation mentale}. Un souvenir, une hypothèse, une intention sont des exemples de représentations mentales. L'utilisateur et le producteur d'une représentation mentale ne font qu'un. Une représentation peut aussi exister dans l'environnement de l'utilisateur comme par exemple le texte qui est sous vos yeux. Il s'agit alors d'une \emph{représentation publique}. Une représentation publique est généralement un moyen de communication entre un producteur et un utilisateur distincts l'un de l'autre.
``%
A representation may exist inside its user: it is then a \emph{mental representation}, such as a memory, a belief, or an intention. The producer and the user of a mental representation are one and the same person.  A representation may also exist in the environment of its user, as is the case, for instance, of the text you are presently reading: it is then a \emph{public representation}.}%~(\citealp[p.~49]{sper:expl}.)
''
%\end{quotation}
}
The key point here is that representations are not being replicated through a hi-fidelity copying process, but are being reformulated through an interpretation process, and thus evolve. Cultural epidemiology postulates this type of conceptual mutation and suggests that it can be appraised through the notion of ``cultural attractor'', seen as the attraction domains of an underlying semantic dynamical system.
Despite some recent modeling attempts (for instance \citet{Claidiere07}), the development of quantitative measurements focused on the key notion of cultural attractor has remained a relatively hard task and, to our knowledge, this hypothesis has not been empirically validated.

%Bien qu'ayant connu des développements récents en modélisation (par exemple \citealp{Claidiere07}, pour la notion d'attracteur culturel), la difficulté à effectuer des mesures quantitatives sur les notions clés du domaine a fait que les hypothèses à la base de l'édifice théorique développé n'ont pas encore pu être testées empiriquement.

The last decade has however witnessed an avalanche of observable \emph{in vivo} data stemming from online interactions. These large amounts of textual traces of information production by humans are plausibly not records of strictly ``physical'' inter-individual interactions (in the sense of ``real life'' interactions), yet they actually constitute a wealth of observations on the dynamics of public representations. As such, they are prone to radically improve the prospects of empirical validation of the individual-level processes of cultural evolution. 

We aim to empirically describe the transformation of a specific type of public representations by focusing on the possible cognitive alterations performed by individuals during reformulation.  % On s'est proposé d'étudier de façon \emph{empirique} la transformation des représentations publiques, en se focalisant sur la mise en évidence d'un biais cognitif dans le traitement  par le cerveau de certaines représentations culturelles. 
To deal with robust and simple cultural representations, we paid attention to the evolution of quotations. Admittedly, these verbatim public representations should not suffer any reformulation alteration (as opposed to more sophisticated expressions and opinions), but still do.  We will in particular exhibit a non-trivial cognitive alteration by which some portions of the quotation, namely individual words, may be modified; and will wonder which are generally the semantic and structural characteristics of these words and their substitutions. 
%On verra que par cette analyse de la reproduction des citations par les auteurs on observe un biais cognitif non trivial ; on observera ainsi un lien entre l'attractivité de certains mots (en termes d'épidémiologie culturelle) et les propriétés sémantiques de ces mêmes mots, calculées sur la base du réseau formé par ces mots.
%On s'est proposé d'étudier de façon \emph{empirique} la transformation des représentations publiques, c'est-à-dire d'explorer à l'aide de données obtenues in vivo l'effet du cerveau sur l'évolution des représentations ; une telle étude permettrait d'observer empiriquement l'effet d'un attracteur culturel tel que défini par Sperber. La question est donc la suivante : peut-on observer un biais cognitif dans le traitement de certaines représentations culturelles par le cerveau, et si oui quel est-il ?On verra que par cette analyse de la reproduction des citations par les auteurs on observe un biais cognitif non trivial ; on observera ainsi un lien entre l'attractivité de certains mots (en termes d'épidémiologie culturelle) et les propriétés sémantiques de ces mêmes mots, calculées sur la base du réseau formé par ces mots.
Broadly, we contend that using this type of data is both equivalent to a large-scale psycholinguistic experiment and at the same time constitutes the first step towards the design of empirically realistic models of cultural evolution. 

% --------------------------------
%\subsection*{Problématique} % -----
% --------------------------------
\label{subsec:problematique}

%\paragraph*{Plan}

The next section (Sec~\ref{sec:related}) describes the relevant recent state-of-the-art on this matter. In Sec.~\ref{sec:protocol}, we detail the empirical protocol and the various assumptions that were made in order to deal with the available empirical material. Section~\ref{sec:results} describes the significant psycholinguistic cues and biases observed during \emph{in vivo} quotation reformulation, followed by a discussion and general guidelines for further work in Sec.~\ref{sec:conclusion}.
%Le travail de recherche effectué étant essentiellement empirique, on le présente en deux grandes étapes : en premier lieu, on détaille le protocole d'expérimentation utilisé ainsi que les différentes tentatives et pistes explorées pendant son élaboration. Pour cela, on commencera par analyser la question ci-dessus plus à fond afin d'en distiller une question opérationnelle, puis on expliquera en détail la façon dont on a construit le protocole expérimental permettant de faire les mesures pour répondre à cette question%\footnote{Le terme "protocole expérimental", utilisé ici, est bien le terme approprié pour ce dont il s'agit : on verra qu'on part d'un corpus de données pour lequel on construit une façon de faire les mesures qu'on considère pertinentes, et cette démarche correspond à la construction d'un protocole expérimental comme on peut le faire pour une expérience de psychologie cognitive ; c'est à ce moment qu'on définit ce qu'on va mesurer et comment cette mesure est faite.}
%. La deuxième grande partie concerne les résultats : on montre tout d'abord la façon dont les mesures sont utilisées pour construire les observables qu'on cherche, ce qui permettra ensuite d'interpréter les résultats obtenus au regard de la question posée.
%Une dernière partie concernera les perspectives que ce travail ouvre, et les façons dont il pourrait être poursuivi et étendu.

\section{Related work}\label{sec:related}

The relevant literature on \emph{public representation dynamics} features two main streams. On one hand, we find studies aiming at understanding the macroscopic phenomena of \emph{social diffusion} of public representations, describing for instance the propagation of cultural items across social networks such as blogspace \citep{Gruhl04}, the characteristic times and diffusion cycles both within these social networks and with respect to the topical dynamics of news media \citep{Leskovec09}, or the reciprocal influence between the social network topology and the distribution of issues \citep{Cointet09}.     
%L'état de l'art en ce qui concerne l'épidémiologie culturelle peut être résumé ainsi : d'un côté, des études %s'intégrant au deuxième groupe de disciplines
% permettent une bonne compréhension des phénomènes macroscopiques de \emph{diffusion sociale} des représentations publiques, tels que le chemin emprunté par une information atomique donnée, les temps caractéristiques et les cycles de diffusion dans ces réseaux, la relation entre blogs et sites de médias, ou encore l'autorité et l'influence exercée par certains n\oe{}uds du réseau. 
These studies are relatively independent from anthropology and cognition and are at the interface between data mining, complex systems and quantitative sociology (first and foremost social network analysis). Without necessarily relying on specific social science theories, %these studies are interesting in that they aim at describing cultural dynamics by using large social media corpuses. 
this research stream is of interest for its aim of cultural dynamics description by using large social media corpora.
% ,    Ces études sont relativement indépendantes de l'anthropologie et de la cognition et  se situent à l'interface entre le \emph{data mining}, l'étude des systèmes complexes, et la sociologie quantitative (e.g. sociologie des réseaux), en exploitant de grands corpus de données produits par des collectes sur Internet. Sans nécessairement s'appuyer sur des cadres théoriques précis ou intégrés, ces études présentent néanmoins la particularité de vouloir décrire les dynamiques culturelles en exploitant les possibilités ouvertes par les larges corpus de données issus d'Internet. %, allant de l'étude de l'impact des réseaux sociaux sur la diffusion des informations sur ces mêmes réseaux \citep{Bakshy12} à l'analyse du rôle de la presse en ligne dans la vie politique locale \citep{Parasie12}, en passant par le détail des interactions entre éditeurs sur Wikipedia \citep{Jurgens12} ou l'évolution du vocabulaire des langues au cours du temps \citep{Lieberman07}. 
%On peut citer ici l'étude de la propagation et de la diffusion de l'information dans la blogosphère faite par \citet{Gruhl04}, l'étude du \emph{news cycle} faite par \citet{Leskovec09} qui fait intervenir la plupart de ces aspects, ou encore l'analyse des influences réciproques entre un réseau social et le réseau sémantique formé par les contenus diffusés sur ce premier réseau, faite par \citet{Cointet09} ; %on l'a déjà mentionné, 
 %, ces thèmes étant explorés principalement par le premier groupe de disciplines. 
 
On the other hand, the study of the \emph{transformation} of public representations has emerged only recently. For one, the notion of ``cultural attractor'' has been the focus of models rendering the evolution of some representations and behaviors not until a few years ago \citep{Claidiere07}. Among the empirical approaches on the mutation of representations, some of the most relevant studies to date consist of a series of papers investigation \emph{quotation} transformations from a large corpus of US blog posts, initially collected and studied by \citet{Leskovec09} and further analyzed by \citet{Simmons11} and \citet{omod-mult}. They generally show several types of regularities and propose models of diffusion-transformation of quotations, which may nonetheless appear to be relatively simplistic from a cognitive viewpoint.  One of the main conclusions of these works lies in the fact that even for the supposedly most stable type of public representation, namely quotations, it is indeed still possible to observe and measure significant transformations.  However, these studies address transformations by focusing on the properties of the author (e.g. news vs. blogs) of the quotation, or the surrounding public space (e.g. quote frequency in the corpus), rather than the very cognitive features which may trigger or, at least, bias these transformations. % Cependant ces auteurs n'étudient les transformations observées qu'en fonction du type d'auteur qui reproduit les citations, et pas en fonction des citations elles-mêmes (en particulier les questions d'ordre cognitif concernant la raison pour laquelle ces transformations ont lieu, ou la raison pour laquelle elles ont lieu dans tel sens et pas dans tel autre, ne sont pas abordées). %Les possibilités offertes par cette étude étant vastes, on a décidé de travailler sur les mêmes données que \citeauthor{Simmons11} pour traiter notre question.

At this level, we have to turn to the broader psycholinguistic literature which provides one of the main cognitive foundations of public representation evolution by studying the influence of word features on the ease of recall. This state-of-the-art is relatively developed and shows the positive impact of classical psycholinguistic variables, such as word frequency (see \citet{Yonelinas02} for a review), age-of-acquisition \citep{Zevin02}, number of phonemes and number of syllables (see for instance \citet{Rey98} and \citet{nick-diss}).

\TB{The study of network-based properties, while heavily discussed around the notion of connectionism and its normative processual models, appeared much more recently as an empirical investigation field \CN} \rk{C'est sûr ? e.g. http://tiny.cc/bl7usw}. Let us mention two interesting and recent studies on that matter, which demonstrate in a strictly \emph{in vitro} framework and at the vocabulary level that word properties computed on a word network are important factors in the cognitive processes and reproduction of those words. %Les deux autres travaux viennent de la psycholinguistique, et posent les fondations pour l'étude des propriétés des représentations (qui seront ici des citations). 
First, \citet{Griffiths07} analyze a task where patients are asked to speak out the first word which comes to their mind when they are presented with a random alphabet letter. The authors show that there exists a link between the ease of recall of words and one of their semantic features, namely their authority position (PageRank) in a language-wide semantic network built from external word association data. % property concernant une tâche où l'on présente aux sujets une lettre de l'alphabet en leur demandant de dire le premier mot commençant par cette lettre qui leur vient à l'esprit. Les auteurs montrent qu'une certaine propriété des mots (le PageRank calculé sur un réseau sémantique construit à partir de données d'associations de mots) est un bon facteur prédictif de quel mot sera rappelé parmi les mots commençant par la lettre demandée ; ils montrent ainsi un lien entre une caractéristique sémantique quantitative des mots et la facilité de rappel de ces mots dans cette tâche. 
A second psycholinguistic study by \citet{Chan10} shows, in a task of picture naming, that words come quicker when they have a higher clustering coefficient in an underlying phonological network (which, again, is  defined from external phonological data). %, étudie une tâche où l'on demande aux sujets de nommer oralement une image. Les auteurs montrent qu'une autre propriété des mots, phonologique cette fois (le coefficient de regroupement des mots calculé sur le réseau phonologique formé par ces mots), a un effet sur la rapidité à nommer l'image présentée : c'est un autre cas de lien entre une caractéristique quantitative (phonologique) d'un mot et la facilité à produire ce mot.

On the whole, the current psycholinguistic state-of-the-art seems to hint at two antagonistic types of results.  On one hand, part of the literature tends to show that recall is easier for the least ``awkward'' words; those whose age of acquisition is earlier, length is smaller, semantic network position is more central -- this is particularly true in tasks where participants are asked to form spontaneous associations or utter a word in response to a given signal \CN. On the other hand, when the task consists in remembering a specific list of items, ``awkward'' words are actually more easily remembered, possibly as they are more informative and plausibly more discernable \CN. The jury is still out as to whether reformulation alteration, i.e. spontaneous replacement of words when asked to repeat a given utterance, is rather of the former or latter sort.  Our paper additionally sheds light on this debate.

\smallskip
More broadly, we thus observe a gap between, on one side, macro-level empirical studies on diffusion dynamics in a social system and, on the other side, studies essentially focused on micro-level transformations of representations --- these latter studies being either strongly normative, or yielding information whose integration into realistic cultural epidemiology models remains difficult. %On remarque de plus que le volet cognitif, central pour les phénomènes étudiés, est relativement peu développé dans les modèles et peu exploré de façon empirique.
%Les deux dernières études, issues de la psycholinguistique et dans un cadre strictement \emph{in vitro} et strictly at the vocabulary level, montrent donc que des propriétés des mots du vocabulaire, calculées sur un réseau formé par ces mêmes mots, sont des facteurs importants dans la façon dont on les traite cognitivement et dans la façon dont on les produit. C'est justement l'élément qui nous manquait : en se basant sur ces trois travaux, on va pouvoir utiliser des propriétés ainsi calculées sur les mots des citations pour mesurer quantitativement les transformations subies par ces citations lorsqu'elles sont reproduites. Pour rendre l'analyse possible, on se concentrera sur des cas de \emph{substitution} d'un mot par un autre lorsqu'un auteur reproduit une citation : à chaque substitution observée, on assiste à un échange entre deux mots pour lesquels on va pouvoir comparer les propriétés sémantiques.



%%% GARBAGE 


%Cependant, depuis plus d'une décennie le paysage des choses observables dans le domaine des représentations publiques est en train de changer radicalement, du fait du déluge de données provenant des interactions en lignes, permettant l'étude empirique des représentations dans de cadre de l'épidémiologie culturelle. 

%Nous vivons aujourd'hui non pas avec un manque de données concernant les représentations publiques, mais au contraire sous un véritable déluge de telles données. Quel est ce déluge ? Il est le produit d'une collecte qui est née avec l'utilisation commerciale et massive d'Internet. Cette collecte n'est pas un phénomène mineur, ni même un phénomène d'ampleur moyenne : les services sur Internet enregistrent aujourd'hui la quasi-totalité de ce qui est mesurable des interactions entre les personnes connectées ainsi que des interactions des personnes avec les machines. Vu l'importance que ce réseau a pris dans nos vies de tous les jours, on assiste bien à une explosion de ces données ; à tel point que certains observateurs d'Internet estiment que 90\% des données stockées aujourd'hui par les services sur Internet ont étés générées il y a moins de deux ans.
%Sans être des enregistrements des interactions \emph{physiques} entre les personnes (``real life''), il s'agit bien d'observations sur le terrain nouveau que constituent l'action, la perception, et l'interaction connectées, au travers des machines et d'Internet. En un mot, elles sont en fait des observations en grandes quantités de \emph{représentations publiques}. %\footnote{Bien sûr, une bonne partie des données collectées sur Internet appartient à des entreprises qui n'ont aucun intérêt à les partager ni à les exploiter autrement que financièrement. Mais les données publiquement accessibles forment tout de même une base considérable, qu'il est tout à fait possible d'exploiter comme telle.}.
%L'approche qu'on a esquissée ici n'est qu'une des façons d'aborder l'étude de la culture et de la cognition de façon empirique. En effet, les courants de recherche qui gravitent autour de ce thème sont nombreux et variés ; voici une rapide présentation des champs concernés et de la façon dont ils interagissent.

%Un groupe de disciplines est particulièrement pertinent ici, à l'interface entre le \emph{data mining}, l'étude des systèmes complexes, et la sociologie quantitative (e.g. sociologie des réseaux). Un grand nombre de travaux apparaissent exploitant de grands corpus de données produits par des collectes sur Internet, allant de l'étude de l'impact des réseaux sociaux sur la diffusion des informations sur ces mêmes réseaux \citep{Bakshy12} à l'analyse du rôle de la presse en ligne dans la vie politique locale \citep{Parasie12}, en passant par le détail des interactions entre éditeurs sur Wikipedia \citep{Jurgens12} ou l'évolution du vocabulaire des langues au cours du temps \citep{Lieberman07}. Une partie de ces travaux n'inscrivent leurs résultats que dans des cadres théoriques naïfs (autour de thèmes pour lesquels il existe pourtant des théories réfléchies développées par des champs des sciences sociales), et relèvent parfois plus des sciences de l'informatique que de l'étude de la culture (d'aucuns parlent d'ailleurs de \emph{social computing}). Ce groupe de disciplines présente néanmoins la particularité de vouloir étudier la culture en exploitant les possibilités ouvertes par les larges corpus de données issus d'Internet.

%% ----------------------------------------
%\subsection{Relevant fields} % -----
%% ----------------------------------------
%
%L'idée d'étudier la culture en lien avec la cognition apparaît comme une extension naturelle pour plusieurs disciplines historiquement séparées qui, chacune de leur côté, s'en sont approchées en partant de leur propre espace de concepts et en étendant leurs outils. Les dialogues entre ces disciplines ne sont pas rares, mais les traditions et les épistémologies différentes ont rendu les unifications plus ardues. On peut distinguer au moins deux grands groupes de disciplines débattant autour de thématiques reliées à ce sujet.%\footnote{On regroupe ici les disciplines par les thèmes qu'elles débattent, plutôt que par les similarités entre leurs façons de traiter ces thèmes.}.
%
%Le premier groupe est issu des débats entre la biologie évolutionniste, la psychologie cognitive et les sciences sociales (et plus particulièrement l'anthropologie), desquels ont émergé les domaines de la psychologie évolutionniste et de l'anthropologie cognitive. %Plusieurs théories ont été avancées et sont débattues en parallèle par ces disciplines. On se doit de mentionner ici en premier lieu le débat autour du programme mémétique initié par \citet{Dawkins76} et pour lequel l'ouvrage collectif de \citet{Aunger00} est un bon exemple. Un deuxième terrain exploré par cet ensemble de disciplines est le développement de modèles évolutionnistes basés sur les normes (voir par exemple \citealp{Ehrlich05}), avec comme travail fondamental celui de \citet{Boyd85}. Le troisième champ, particulièrement important aujourd'hui, est précisément celui de l'épidémiologie culturelle, initié par \citet{Sperber96} et défendu face aux autres courants dans des travaux comme celui de \citet{Atran03}.
%
%Le deuxième groupe, plus diffus, provient des échanges entre le \emph{data mining}, l'étude des systèmes complexes, et la sociologie quantitative (e.g. sociologie des réseaux). Un grand nombre de travaux apparaissent exploitant de grands corpus de données produits par des collectes sur Internet, allant de l'étude de l'impact des réseaux sociaux sur la diffusion des informations sur ces mêmes réseaux \citep{Bakshy12} à l'analyse du rôle de la presse en ligne dans la vie politique locale \citep{Parasie12}, en passant par le détail des interactions entre éditeurs sur Wikipedia \citep{Jurgens12} ou l'évolution du vocabulaire des langues au cours du temps \citep{Lieberman07}. Une partie de ces travaux n'inscrivent leurs résultats que dans des cadres théoriques naïfs (autour de thèmes pour lesquels il existe pourtant des théories réfléchies développées par des champs des sciences sociales), et relèvent parfois plus des sciences de l'informatique que de l'étude de la culture (d'aucuns parlent d'ailleurs de \emph{social computing}). Ce groupe de disciplines présente néanmoins la particularité de vouloir étudier la culture en exploitant les possibilités ouvertes par les larges corpus de données issus d'Internet.


