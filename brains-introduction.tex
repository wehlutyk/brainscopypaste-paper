% !TEX root = brainscopycut.tex

% ============================
\section{Introduction} % =====
% ============================

\todo{\#20: check text/flow/definitions for clarity against Gureckis' edited pdf}

\begin{new}
START SENTENCE.

Since the very beginnings of both social science and psychology, scientists have tried to capture the way cognition and culture influence each other.
%the reciprocal influences of cognition and culture in meaningful ways.
While it has been the subject of intense debate in the social sciences in the 20th century (starting with Durkheim's initial works, \citeyear{durkheim_les_1912}, later tackled in earnest by e.g. Mauss' \emph{Techniques of the Body}, \citeyear{mauss_sociology_1979}, Giddens' \emph{Structuration Theory}, \citeyear{giddens_constitution_1984}, and Bourdieu's \emph{Sens Pratique}, \citeyear{bourdieu_sens_1980}), today's discussion is mostly structured by proponents from cognitive science.

These construe culture as an evolutionary process analogous and parallel to biological evolution (and especially the modern synthesis' account of it).
That analogy can be traced a long way back in the 20th century and earlier, with milestones such as Kroeber's works~\citeyearpar{kroeber_nature_1952}, Dawkins' \emph{Memetics}~\citeyearpar{dawkins_selfish_2006}, and later the development of \emph{Dual inheritance theory} by \citet{boyd_culture_1985} and \citet{cavalli-sforza_cultural_1981} among others.
More recently, Dan Sperber has drawn on this principle to explicitly connect anthropology and cognitive science through the theory of \emph{Epidemiology of Representations} \citep{sperber_explaining_1996}, and the study of cultural evolution has been growing steadily since.

The collection of works by \citet{aunger_darwinizing_2000} (in particular \citealp{bloch_well-disposed_2000}, and \citealp{kuper_if_2000}) have shown how the theory of memetics cannot account for the levels of transformation culture undergoes as it is transmitted.
\Citet{mesoudi_multiple_2008} have discussed the uses of transmission chain experiments to test what dual inheritance theory can explain about cultural evolution.
\Citet{morin_how_2013} and \citet{miton_universal_2015}, by carefully compiling a series of anthropological works, show how cognitive biases have influenced the evolution of cultural artifacts over several centuries.
\Citeauthor{kirby_cumulative_2008} \citeyearpar{kirby_cumulative_2008,cornish_systems_2013} have shown how evolutionary pressures lead to the emergence of structured and expressive artificial languages in simulations and laboratory experiments.
Such transmission chain experiments have also been explored in non-human primates by \citet{claidiere_cultural_2014}.

The theory of epidemiology of representations proposes a unifying framework for all these works by recasting them as questions of spread and transformation of representations:
these are alternatively located in the mind ("mental representation" in Sperber's terminology), or in the outer world ("public representations") as expressions of mental representations in diverse cultural artifacts (pieces of text, utterances, pictures, building techniques, etc.).
A human society is then modeled as a large dynamical system of people constantly interpreting public representations into mental representations, and producing new public representations based on what they have previously interpreted.
Two key points are that (a) transmission is not reliable (representations change significantly each time they are interpreted and produced anew, as opposed to e.g. memetics), and (b) the reciprocal influences of cognition and culture can be captured by studying the evolution of public representations themselves, which is what the studies cited above are doing.

The theory makes an additional strong hypothesis, which this paper focuses on:
the accumulation of cognitive effects in transmission leads to the emergence of very stable representations (called "cultural representations") that spread throughout an entire society, a process which should manifest itself as attractors (called "cultural attractors") in the dynamical system that models that society.

This hypothesis has been hard to test in concrete situations, as quantitative data on out-of-laboratory cultural artifacts is not easy to collect.
One approach, as mentioned above, has been the meta-analysis of large bodies of anthropological studies (see \citealp{miton_universal_2015}, for instance).
This paper exemplifies a second approach, taking advantage of the ever-increasing avalanche of available digital footprints since the 2000's.
Indeed, tools and computing power to analyze such data are now widespread, and the body of research aimed at describing online communities and content is growing accordingly (see for instance \citealp{gruhl_information_2004}, \citealp{liben-nowell_tracing_2008}, \citealp{cointet_socio-semantic_2009}, and \citealp{danescu-niculescu-mizil_you_2012}, for good examples).
We believe these tools can connect the field of cultural evolution with psycholinguistics to advance the testing of cultural attractors.

\bigskip

To do so we analyze the way quotes in blogs and media outlets are modified when they are copied from website to website.
These public representations should normally not change when they are copied from one website and published on another (as opposed to more elaborate expressions or opinions, not identified as quoted utterances), but empirical observation shows that they are in fact occasionally transformed \citep{simmons_memes_2011}.\footnote{
It seems that authors do not always copy-paste, but copy quotes by hand instead.
}
Analyzing this process gives us information on (a) the way these public representations are transformed when they are transmitted, and (b) which cognitive biases are responsible for those transformations.
By using an exploratory approach, our study gives a detailed view of one of the cognitive effects responsible for the evolution of online quotes, and shows that it is compatible with the hypothesis of cultural attractors.

We chose to restrict our analysis to substitutions (one word being replaced by another), both to keep the analysis tractable and because of missing information in our data set.\footnote{
As explained further down, source-destination links between quotes must be inferred from the data set, an operation which is much more reliable if we restrict our analysis to substitutions.
This also impedes us from observing the effect of accumulated transformations in the long term, limiting our results to a view of the individual evolutionary step.
}
While this restricts the scope of our observations for this particular data set, the methodological point we also make is left intact.

By characterizing substitutions with 6 features on the disappearing and appearing words, we identify what makes a substitution more likely, and how a word changes when it is substituted.
Consistent with known effects in linguistics, we observe that low-frequency words and words learned later in development are more susceptible to substitution than other words.
Looking at the context those words appear in, we observe a marked effect for substitution of extreme words in a sentence (either very high-valued or very low-valued features compared to sentence average, except for word frequency).
Focusing on how words are transformed, we see that the appearing words have significantly higher frequency and lower age-of-acquisition than synonyms of the disappearing word.
Finally, the patterns we observe are also consistent with an attraction of each of the features towards a (feature-specific) asymptotic value.

While the transformations we describe are not the only ones at work in this data set, our analysis shows that the substitution process is both consistent with known biases and compatible with the existence of feature-specific attractors.
This study can be viewed as analyzing part of the transmission step operating in transmission chains of artificial languages like those studied by \citet{kirby_cumulative_2008}, but with natural language out of the laboratory.

The next section %(Sec.~\ref{sec:related})
describes our hypotheses along with a review of the psycholinguistics literature.
%In Sec.~\ref{sec:protocol},
Then, we describe the data set and detail the various assumptions that were made in order to analyze it.
%Sec.~\ref{sec:results}
Next, we describe the measures built to observe the cognitive biases operating in quote transmission.
Finally, we discuss the relevance of these results for the study of cultural evolution, followed with general guidelines for further work.%Sec.~\ref{sec:conclusion}.

\end{new}

The understanding of knowledge transmission mechanisms has led to a sizable literature in the recent past, spanning over numerous research fields ranging from cultural anthropology to social network analysis and complex systems modeling; from social cognition to data mining. These works are diversely labeled as studies on ``opinion dynamics'', ``cultural evolution'', or ``information diffusion'' and, for the most part, investigate phenomena pertaining to both cognitive science and social science, both at the individual and social levels.

Broadly speaking, we may distinguish two main research streams, depending on whether the focus lies on cognitive processes or on social dynamics. A first stream is largely structured around cultural anthropology and essentially addresses cultural similarity, diversity and its evolution. It features several theories mixing social and individual cognition including, to cite a few, the debated ``memetic'' program initiated by \citet{Dawkins76} (for which the collection of works by \citealp{Aunger00}, provides a solid overview), the development of evolutionary models of norms~\citep[see for instance][]{Ehrlich05} following the seminal work of \citet{Boyd85}; or the ``cultural epidemiology'' program proposed by \citet{sper:expl}, which links the concept of mental representation to the concept of public representation (the latter being the counterpart of the former outside of the brain, \hbox{i.e.} in all kinds of cultural artifacts: texts, utterances, etc.).
%\footnote{\citet{sper:expl} emphasizes this distinction in his seminal work: ``A representation may exist inside its user: it is then a \emph{mental representation}, such as a memory, a belief, or an intention. The producer and the user of a mental representation are one and the same person. A representation may also exist in the environment of its user, as is the case, for instance, of the text you are presently reading: it is then a \emph{public representation}''.}

One of the core claims of this literature consists in emphasizing that not all knowledge is equally fit for being reproduced, although the various approaches have a different take on how exactly this notion of fitness should be operationalized. Sperber's cultural epidemiology classically opposes Dawkins' memetics by insisting that representations are not being replicated through a high-fidelity copy process, but are being interpreted and produced anew, and are thus greatly subject to change.
Cultural epidemiology postulates that this conceptual evolution can be appraised through the notion of ``cultural attractor'', seen as the attraction domain of an underlying socio-semantic dynamical system.\footnote{Works such as \citet{Atran03} argue that this approach is anthropologically better suited than memetics, and some of the main issues in this debate are further detailed by \citet{Kuper00} and \citet{Bloch00}.}
Despite some recent modeling attempts~\citep[e.g.][]{Claidiere07}, the development of quantitative measurements relying on the concept of cultural attractors has remained a relatively hard task and, to our knowledge, this hypothesis has not yet been empirically analyzed in an extensive manner.


Another research stream deals with rather macroscopic studies of knowledge diffusion. Here, one of the focal points is that not all knowledge gets propagated identically along the same routes, within the same communities, at the same pace. The various approaches usually feature a minimalistic description of cognitive processes, strongly reminiscent of biological epidemiology (a single, atomic piece of information may or may not be adopted by each individual).
This research program nonetheless exhibits a particularly interesting empirical track record --- largely owing to a recent avalanche of observable \emph{in vivo} data which, for a good decade now, have mainly come from online interaction contexts.
While these information trails are not records of ``physical'' inter-individual interactions (in the sense of ``real life'' interactions), they still constitute a wealth of observations on the dynamics of public --~albeit online~-- representations.
Some authors could describe for instance the propagation of cultural artifacts across social networks such as blogspace~\citep{Gruhl04} or the email network~\citep{Liben-Nowell25032008}, the characteristic times and diffusion cycles both within these social networks and with respect to the topical dynamics of news media~\citep{Leskovec09}, or the reciprocal influence between the social network topology and the distribution of issues~\citep{Cointet09}.

These latter studies are at the interface between data mining, complex systems and quantitative sociology (first and foremost social network analysis) and are relatively remote from cognitive science; for a significant part, they rely rather marginally on specific social science theories. They nonetheless show us the added value of using these rapidly growing records %--~given the already significant and rapidly growing importance of our online interactions~--
%and, %more broadly, large textual corpora \cam{google n-grams??}
towards radically improving the empirical understanding of (individual-level) cultural evolution processes.


\bigskip
Stepping back, we thus observe a gap between, on one side, empirical studies of diffusion dynamics in social systems and, on the other side, more theoretical works  focused on knowledge transformation processes. %--- these latter studies being either strongly normative, or with results difficult to articulate with realistic cultural epidemiology models.
Our research lies at the intersection of these two programs, aiming to shed light on micro-level information transformation by leveraging the empirical wealth of (\emph{in vivo}) social diffusion phenomena. More precisely, we hope to describe reformulation processes within a large distributed system such as blogspace; showing how some specific types and features of public representations may be altered by bloggers when they freely reproduce them.

We focus on simple linguistic modifications, thereby connecting our research to the broader psycholinguistic literature.
To deal with robust and simple cultural representations, we paid attention to the evolution of quotations.
While these verbatim public representations should in theory not suffer any alterations when they are produced anew (as opposed to more elaborate expressions and opinions, not identified as quoted utterances), empirical observation shows that they are occasionally transformed.
We will in particular exhibit a non-trivial process by which individual words in quotations are replaced.
We will uncover some of the semantic and structural characteristics of these words and the substitutions they undergo.
In a way using this type of data is equivalent to a large-scale psycholinguistic experiment and at the same time constitutes a first step towards building empirically realistic models of cultural evolution.

The next section %(Sec.~\ref{sec:related})
describes our hypotheses along with the relevant state-of-the-art on this psycholinguistic matter.
%In Sec.~\ref{sec:protocol},
Then, we detail the empirical protocol and the various assumptions that were made in order to deal with the available empirical material.
%Sec.~\ref{sec:results}
We further describe the significant psycholinguistic biases observed during \emph{in vivo} quotation reformulation as well as their epidemiological setting, followed by a discussion and general guidelines for further work in the final section.%Sec.~\ref{sec:conclusion}.


% ============================
\section{Related work} % =====
% ============================
\label{sec:related}

\todo{
\#15: add
(1) sentence recall (Potter \& Lombardi),
(2) false memories (Deese),
(3) subjective organization (Tulving, Zaromb),
(4) working memory and attention (Jefferies),
(5) iterated learning (Kirby)
}

\todo{\#20: check text/flow/definitions for clarity against Gureckis' edited pdf}

%The relevant literature on \emph{public representation dynamics} features two main streams.
%On one hand, we find studies of the macroscopic \emph{social diffusion} of public representations, describing for instance the propagation of cultural artifacts across social networks such as blogspace \citep{Gruhl04}, the characteristic times and diffusion cycles both within these social networks and with respect to the topical dynamics of news media \citep{Leskovec09}, or the reciprocal influence between the social network topology and the distribution of issues \citep{Cointet09}.
%These studies are relatively independent from anthropology and cognition and are at the interface between data mining, complex systems and quantitative sociology (first and foremost social network analysis).
%Without necessarily relying on specific social science theories, this research stream is of interest for its use of large textual corpora in studying cultural dynamics.

The practical study of the transformation of public representations has emerged only recently.
For one, models involving evolution and representations to study the notion of ``cultural attractor'' have appeared only a few years ago (see \citealp{Claidiere07}, and \citealp{claidiere2014darwinian}, as well as a hybrid empirical-theoretical protocol in \citealp{maccallum2012evolution}).
Among the empirical approaches, some of the most relevant studies to date consist in a series of papers investigating \emph{quotation} transformations in a large corpus of US blog posts, initially collected and studied by \citet{Leskovec09} and further analyzed by \citet{Simmons11} and \citet{omod-mult}.
%They exhibit several types of regularities and propose diffusion-transformation models of the evolution of quotations, which may nonetheless appear to be relatively simplistic from a cognitive viewpoint. One of the main observations in these works is that even for quotations, a type of public representation that should be among the most stable, it is still possible to witness significant transformations. However, these studies address transformations by focusing on the properties of the source of the quotation (\hbox{e.g.} news outlet {vs.} blog), or the surrounding public space (\hbox{e.g.} quotation frequency in the corpus), rather than the very cognitive-level features which may determine or, at least, influence these transformations.
One of the main observations in these works is that even for quotations, a type of public representation that should be among the most stable, it is still possible to witness significant transformations. They essentially examine the effect of some properties of the quotation source (\hbox{e.g.} news outlet {vs.} blog) or of the surrounding public space (\hbox{e.g.} quotation frequency in the corpus). Some diffusion-transformation models have been proposed, yet the very cognitive features which may determine or, at least, influence these transformations, are overlooked; which may appear to be relatively unsatisfying from a cognitive viewpoint.


At this level, we have to turn to the broader psycholinguistic literature which provides one of the main cognitive foundations for public representation evolution by studying the influence of word features on the ease of recall.
This field is well developed and details the impact that classical psycholinguistic variables such as word frequency~\citep[see][for a review]{Yonelinas02}, age-of-acquisition~\citep{Zevin02}, number of phonemes or number of syllables~\citep[see for instance][]{Rey98,nick-diss}, have on this type of task.

Less classical linguistic variables, based on the study of semantic network properties, have recently started to be used, in the context of connectionism and its normative processual models~\citep[see for instance][]{collins1975spreading}.
Let us mention four interesting studies on that matter, which demonstrate in a strictly \emph{in vitro} framework and at the vocabulary level that properties computed on a word network are important factors for the cognitive processes and reproduction of those words.
First, \citet{Griffiths07} analyze a task where subjects are asked to name the first word which comes to their mind when they are presented with a random letter from the %Latin
alphabet. The authors show that there exists a link between the ease of recall of words and one of their semantic features, namely their authority position (pagerank) in a language-wide semantic network built from external word association data.
\citet{austerweil2012human} further develop this idea by showing that random walk on such a semantic network, that is the exact process measured by the pagerank index, gives a parsimonious account of some semantic retrieval effects (namely, related items being retrieved together).
A third psycholinguistic study by \citet{Chan10} shows, in a picture-naming task, that words are produced faster and with fewer mistakes when they have a lower clustering coefficient in an underlying phonological network (which, again, is  defined from external phonological data).
\citet{nelson2013activation}, finally, show the importance of clustering coefficient in a semantic network by studying the role it plays in a variety of recall and recognition tasks (extralist and intralist cuing, single item recognition, and primed free assocation).

On the whole, the current psycholinguistic state-of-the-art seems to hint towards two antagonistic types of results.
On one hand, part of the literature tends to show that recall is easier for the least ``awkward'' words; those whose age of acquisition is earlier, length is smaller, semantic network position is more central --- this is particularly true in tasks where participants are asked to form spontaneous associations or utter a word in response to a given signal.
On the other hand, when the task consists in recognizing a specific item in a list, ``awkward'' words are actually more easily remembered, possibly as they are more informative and plausibly more discernible~\citep[see again][for a review]{Yonelinas02}.
The jury is still out as to whether reformulation alteration, that is spontaneous replacement of words when asked to repeat a given utterance, is rather of the former or latter sort.
We also aim here at shedding some light on this debate, considering oddness as a dimension of the purported fitness of utterances.
