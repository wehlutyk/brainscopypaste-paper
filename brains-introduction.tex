% !TEX root = brainscopypaste.tex
% ============================
\section{Introduction} % =====
% ============================

The understanding of the mechanisms behind cultural similarity and diversity has led to a sizeable literature in the recent past, spanning over a vast area of research fields ranging from cultural anthropology to social network analysis and complex systems modelling, all diversely labelled as studies on ``opinion dynamics'', ``cultural evolution'', or ``information diffusion'', for the most part.
Broadly, this type of research program investigates phenomena pertaining to both cognitive science and social science, and aims to understand the processing, transmission, and evolution of information both at the individual and social levels.

Several theories have been proposed and debated within these research fields, mixing social and individual cognition, notably from the side of cultural anthropology.
First, the debate around the ``memetic'' program initiated by \citet{Dawkins76}, for which the collection of works by \citet{Aunger00} provides a solid overview.
A second field has focused on the development of evolutionary models of norms (see for instance \citealp{Ehrlich05}), following the seminal work of \citet{Boyd85}.

A third approach, ``cultural epidemiology'', initiated by \citet{sper:expl} recently attracted significant attention.
Works such as \citet{Atran03} argue that this approach is anthropologically better suited than memetics, and the main issues in this debate are further detailed by \citet{Kuper00} and \citet{Bloch00}.
The cultural epidemiology program focuses on the notion of \emph{representation}, linking the cognitive science concept of \emph{mental representation} to the concept of \emph{public representation}, the latter being the former's counterpart outside the brain (i.e. in cultural artefacts: texts, utterances, etc.).\footnote{\citet{sper:expl} emphasizes this distinction in his seminal work:
\begin{quotation}
A representation may exist inside its user: it is then a \emph{mental representation}, such as a memory, a belief, or an intention.
The producer and the user of a mental representation are one and the same person.
A representation may also exist in the environment of its user, as is the case, for instance, of the text you are presently reading: it is then a \emph{public representation}.
\end{quotation}}
The key point here is that representations are not being replicated through a high-fidelity copy process, but are being interpreted and produced anew, and are thus greatly subject to change.
Cultural epidemiology postulates this type of conceptual evolution and suggests that it can be appraised through the notion of ``cultural attractor'', seen as the attraction domain of an underlying socio-semantic dynamical system.
Despite some recent modelling attempts (for instance \citet{Claidiere07}), the development of quantitative measurements focused on the key notion of cultural attractor has remained a relatively hard task and, to our knowledge, this hypothesis has not yet been empirically analysed.

However, the last decade has witnessed an avalanche of observable \emph{in vivo} data in the form of online interactions.
While they are not records of ``physical'' inter-individual interactions (in the sense of ``real life'' interactions), these productions and information trails still constitute a wealth of observations on the dynamics of public --~albeit online~-- representations.
Given the already significant --~and rapidly growing~-- importance of our online interactions, these records can dramatically improve the prospects of empirical study of the individual-level processes of cultural evolution.

We aim to empirically describe the transformation of a specific type of public representation, by focusing on the possible alterations introduced by individuals when newly producing a representation.
To deal with robust and simple cultural representations, we paid attention to the evolution of quotations.
While these verbatim public representations should in theory not suffer any alterations as they are produced anew (as opposed to more elaborate expressions and opinions, not identified as quoted utterances), empirical observation shows that they are in fact quite often transformed.
We will in particular exhibit a non-trivial process by which individual words in quotations are replaced.
We will uncover some of the semantic and structural characteristics of these words and the substitutions they undergo.
More generally, we contend that using this type of data is equivalent to a large-scale psycholinguistic experiment and at the same time constitutes the first step towards building empirically realistic models of cultural evolution.

The next section (Sec~\ref{sec:related}) describes the state-of-the-art on this matter.
In Sec.~\ref{sec:protocol}, we detail the empirical protocol and the various assumptions that were made in order to deal with the available empirical material.
Section~\ref{sec:results} describes the significant psycholinguistic cues and biases observed during \emph{in vivo} quotation reformulation, followed by a discussion and general guidelines for further work in Sec.~\ref{sec:conclusion}.

% ============================
\section{Related work} % =====
% ============================
\label{sec:related}

The relevant literature on \emph{public representation dynamics} features two main streams.
On one hand, we find studies of the macroscopic \emph{social diffusion} of public representations, describing for instance the propagation of cultural artefacts across social networks such as blogspace \citep{Gruhl04}, the characteristic times and diffusion cycles both within these social networks and with respect to the topical dynamics of news media \citep{Leskovec09}, or the reciprocal influence between the social network topology and the distribution of issues \citep{Cointet09}.
These studies are relatively independent from anthropology and cognition and are at the interface between data mining, complex systems and quantitative sociology (first and foremost social network analysis).
Without necessarily relying on specific social science theories, this research stream is of interest for its use of large social media corpora in studying cultural dynamics.
 
On the other hand, the study of the \emph{transformation} of public representations has emerged only recently.
For one, models involving evolution and representations to study the notion of ``cultural attractor'' have appeared only a few years ago \citep{Claidiere07}.
Among the empirical approaches on the mutation of representations, some of the most relevant studies to date consist in a series of papers investigating \emph{quotation} transformations in a large corpus of US blog posts, initially collected and studied by \citet{Leskovec09} and further analyzed by \citet{Simmons11} and \citet{omod-mult}.
They show several types of regularities and propose diffusion-transformation models of the evolution of quotations, which may nonetheless appear to be relatively simplistic from a cognitive viewpoint.
One of the main conclusions of these works is that even for quotations, a type of public representation that should be among the most stable, it is still possible to observe and measure significant transformations.
However, these studies address transformations by focusing on the properties of the source of the quotation (e.g. news outlet v. blog), or the surrounding public space (e.g. quotation frequency in the corpus), rather than the very cognitive-level features which may determine or, at least, influence these transformations.

At this level, we have to turn to the broader psycholinguistic literature which provides one of the main cognitive foundations for public representation evolution by studying the influence of word features on the ease of recall.
This field is well developed and details the impact that classical psycholinguistic variables such as word frequency (see \citet{Yonelinas02} for a review), age-of-acquisition \citep{Zevin02}, number of phonemes or number of syllables (see for instance \citet{Rey98} and \citet{nick-diss}), have in this type of task.

Less classical linguistic variables, based on the study of semantic network properties, have recently appeared as an empirical investigation field, after having been heavily discussed around the notion of connectionism and its normative processual models (see for instance \citet{collins1975spreading}).
Let us mention two interesting and recent studies on that matter, which demonstrate in a strictly \emph{in vitro} framework and at the vocabulary level that word properties computed on a word network are important factors for the cognitive processes and reproduction of those words.
First, \citet{Griffiths07} analyse a task where patients are asked to name the first word which comes to their mind when they are presented with a random letter from the Latin alphabet.
The authors show that there exists a link between the ease of recall of words and one of their semantic features, namely their authority position (PageRank) in a language-wide semantic network built from external word association data.
A second psycholinguistic study by \citet{Chan10} shows, in a picture-naming task, that words come quicker when they have a higher clustering coefficient in an underlying phonological network (which, again, is  defined from external phonological data).

\rk{add \url{https://www.sciencedirect.com/science/article/pii/S0893608012000330}, \url{http://research.clps.brown.edu/austerweil/pdfs/papers/randomWalkNips2012.pdf}, \url{http://link.springer.com/article/10.3758/s13421-013-0312-y} to that picture}

\TB{On the whole, the current psycholinguistic state-of-the-art seems to hint towards two antagonistic types of results.
On one hand, part of the literature tends to show that recall is easier for the least ``awkward'' words; those whose age of acquisition is earlier, length is smaller, semantic network position is more central -- this is particularly true in tasks where participants are asked to form spontaneous associations or utter a word in response to a given signal \CN.
On the other hand, when the task consists in remembering a specific list of items, ``awkward'' words are actually more easily remembered, possibly as they are more informative and plausibly more discernible \CN.
The jury is still out as to whether reformulation alteration, i.e. spontaneous replacement of words when asked to repeat a given utterance, is rather of the former or latter sort.
Our paper additionally sheds light on this debate.}
\rk{But do we really care about this?}

\smallskip
Stepping back, we observe a gap between, on one side, macro-level empirical studies of the diffusion dynamics in a social system and, on the other side, studies focused on micro-level transformations of representations --- these latter studies being either strongly normative, or whose results are difficult to articulate with realistic cultural epidemiology models.